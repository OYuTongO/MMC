\documentclass[a4paper,12pt]{ctexart}
\usepackage[utf8]{inputenc}
\usepackage{xeCJK}  % 中文支持
\usepackage{amsmath,amsthm,amssymb}  % 数学包
\usepackage{titlesec} % 引入titlesec包来控制标题格式
\usepackage{caption}
\usepackage{graphicx}  % 插入图片
\usepackage{booktabs}  % 美化表格

\usepackage{hyperref}  % 超链接
\usepackage{enumitem}  % 自定义列表
\usepackage[left=3.5cm,right=3cm,top=2cm,bottom=2cm]{geometry}  % 页面设置
\newtheorem{theorem}{定理}[section]

\title{ 商品期货涨跌幅预测问题}
\date{}
\hypersetup{
  hidelinks, % 隐藏链接框
}
% \captionsetup{labelformat=simple}
\renewcommand{\figurename}{}
\renewcommand{\tablename}{}
% 使 section 标题居中
\titleformat{\section}
{\normalfont\Large\bfseries\centering} % 格式设置为居中
{\thesection} % section 标题前的编号
{1em} % 标题和编号之间的间距
{} % 编号后面的格式

\begin{document}

\maketitle
 
% \tableofcontents  % 自动生成目录

 
\section{摘要}


 
\section{问题重述}
\subsection{问题背景}
商品期货(如螺纹钢、铁矿石、焦炭、焦煤等)是金融市场中的重要交易品种,其价格
波动受到多种因素的影响,包括供需关系、宏观经济政策、国际市场变化等。若能利用历
史数据预测商品期货未来的涨跌幅,则可帮助投资者更好地进行交易决策。




\subsection{问题描述}
现有数据集为 1 分钟级数据,包括时间戳、开盘价、最高价、最低价、收盘价、成交
量、持仓量等。请基于该数据集建立数学模型,预测商品期货未来 30 分钟的涨跌幅。涨跌
幅定义为
涨跌幅 = ${ \frac{Pt+30 - Pt}{Pt} * 100\% }$
其中 $P_t$ 是当前时刻的价格,$P_{t+30}$ 是 30 分钟后的价格。。要求从 1 分钟级数据中提取出可
能影响 30 分钟涨跌幅的特征,选择合适的机器学习模型对未来 30 分钟的涨跌幅进行预测。
解释模型的选择理由,并使用适当的评价指标评估模型的性能,讨论模型的局限性及可能
的改进方向。
\section{数据预处理与特征提取}
\section{模型选择}
\section{模型训练与验证}
\section{模型预测效果分析与改进方向}
\section{源码与文档}

\end{document}