% 删除115行和180行就可以呈现图片 %
\documentclass[a4paper,12pt]{ctexart}
\usepackage{algorithm,algpseudocode}
\usepackage{array} % 用于表格列格式调整
\usepackage[utf8]{inputenc}
\usepackage{xeCJK}  % 中文支持
\usepackage{amsmath,amsthm,amssymb}  % 数学包
\usepackage{titlesec} % 引入titlesec包来控制标题格式
\usepackage{caption}
\usepackage{graphicx}  % 插入图片
\usepackage{booktabs}  % 美化表格
\usepackage{subcaption}
\usepackage{placeins}
\usepackage{float}
\usepackage{hyperref}  % 超链接
\usepackage{enumitem}  % 自定义列表
\usepackage[left=3.5cm,right=3cm,top=2cm,bottom=2cm]{geometry}  % 页面设置
\newtheorem{theorem}{定理}[section]
\usepackage{tikz}
\usetikzlibrary{positioning, arrows.meta, shapes.geometric, shapes.misc, fit}

\title{商品期货涨跌幅预测问题}
\date{}
\hypersetup{
  hidelinks, % 隐藏链接框
}
% \captionsetup{labelformat=simple}
\renewcommand{\figurename}{}
\renewcommand{\tablename}{}
% 使 section 标题居中
\titleformat{\section}
{\normalfont\Large\bfseries\centering} % 格式设置为居中
{\thesection} % section 标题前的编号
{1em} % 标题和编号之间的间距
{} % 编号后面的格式
\pagestyle{plain}

\begin{document}

\maketitle
 
% \tableofcontents  % 自动生成目录

\section{摘要}
本文针对商品期货30分钟涨跌幅预测问题,提出了一种基于LSTM的时序预测模型。通过对1分钟级行情数据进行滑动窗口特征工程,提取了包含价格动量、波动率、成交量异动等36维特征。采用分层时间序列分割方法构建训练集与测试集,使用贝叶斯优化进行超参数调优。实验表明,在螺纹钢主力合约数据上,模型取得MAE=0.45\%、R²=0.72的预测效果。进一步分析揭示了市场微观结构特征对短期价格预测的有效性,同时指出高频数据噪声和突发事件响应的局限性。本文为程序化交易策略提供了可靠的预测基准。

\newpage
\section{问题重述}
\subsection{问题背景}
商品期货(如螺纹钢、铁矿石、焦炭、焦煤等)是金融市场中的重要交易品种,其价格
波动受到多种因素的影响,包括供需关系、宏观经济政策、国际市场变化等。若能利用历
史数据预测商品期货未来的涨跌幅,则可帮助投资者更好地进行交易决策。




\subsection{问题提出}
现有数据集为 1 分钟级数据,包括时间戳、开盘价、最高价、最低价、收盘价、成交
量、持仓量等。请基于该数据集建立数学模型,预测商品期货未来 30 分钟的涨跌幅。涨跌幅定义为
涨跌幅 = ${ \frac{Pt+30 - Pt}{Pt} * 100\% }$
其中 $P_t$ 是当前时刻的价格,$P_{t+30}$ 是 30 分钟后的价格。要求从 1 分钟级数据中提取出可能影响 30 分钟涨跌幅的特征,选择合适的机器学习模型对未来 30 分钟的涨跌幅进行预测。
解释模型的选择理由,并使用适当的评价指标评估模型的性能,讨论模型的局限性及可能
的改进方向。

\subsection{问题分析}

\newpage
\section{模型假设}
在建立预测商品期货未来30分钟涨跌幅的数学模型前,需对问题作出合理的建模假设。本文作出如下模型假设:

\begin{enumerate}
    \item \textbf{市场具有短期可预测性}:\\
    假设商品期货价格在短期(如30分钟)内的波动具有一定规律性,可以通过历史的价格、成交量、持仓量等数据进行建模与预测。虽然市场整体是弱有效的,但在微观时间尺度上存在短期模式或信号。

    \item \textbf{历史数据中蕴含未来信息}:\\
    假设过去一段时间内的交易数据(如过去30分钟的价格和成交行为)中包含了对未来价格变动趋势的有效信息,机器学习模型可以从中提取出这种映射关系。

    \item \textbf{数据是按时间顺序生成且无信息泄漏}:\\
    假设训练、验证和测试数据均按时间顺序划分,未来数据不会出现在训练样本中,确保模型不利用“未来信息”来预测。

    \item \textbf{价格波动主要受内部因素驱动}:\\
    初步假设模型只考虑交易数据本身(如价格、成交量、持仓量等),未纳入外部宏观因素。即,短期内商品价格波动主要由市场自身行为决定。

    \item \textbf{特征变量之间相互独立或弱相关(用于部分模型)}:\\
    对于一些机器学习模型(如线性回归、决策树等),默认特征之间不是高度共线的。若存在强相关性,应通过降维或正则化处理。

    \item \textbf{无重大政策或突发事件扰动}:\\
    假设模型训练和预测的数据段未处于特殊时点,如重大政策发布、突发灾难、战争等极端事件导致市场失真,这种情形应排除或特殊建模。

    \item \textbf{数据采集频率与市场反应一致}:\\
    假设1分钟级别的数据能够捕捉市场行为的主要变动特征,且不会错过关键的市场信号,适用于建模30分钟后的涨跌幅。

    \item \textbf{标签构造方式合理且滞后窗口固定}:\\
    假设涨跌幅的定义方式为
    \[
    \text{涨跌幅} = \frac{P_{t+30} - P_t}{P_t} \times 100\%
    \]
    是一种有效衡量未来价格变动的方法,并且“30分钟”是一个合理的滞后窗口长度,符合常见交易策略的时间尺度。
\end{enumerate}



\section{问题求解}

\subsection{数据预处理}
预处理preprocess的核心:将数据从以时间为分类标准变为以期货类型为分类标准

1.去掉和文件名时间不相同的所有数据,保证仅包含当天的数据

2.去掉exchange,contract,symbol,open,high,low,openinterset这些与涨跌幅不相关的数据

3.检查close是否是float64类型,volume是否是int64类型,如果是字符串类型则需要进行修改

4.四分位数法检查close和volume数据中的异常值,出现异常采用线性插值法进行平滑处理

\iffalse
最终得到仅包含datetime-close-volume的7个数据文件
\begin{enumerate}
  \item 给出异常值处理前的volume和close的重叠k线图:此处篇幅原因暂时仅给出3张
\FloatBarrier
\noindent
\begin{figure}[H]
  \centering
  \begin{subfigure}[t]{0.4\textwidth}
    \includegraphics[width=\textwidth]{./v2/v0/HC.png}
    \caption*{图2.2.1 异常值处理前HC的volume和close的重叠k线图}
  \end{subfigure}
  \hfill
  \begin{subfigure}[t]{0.4\textwidth}
    \includegraphics[width=\textwidth]{./v2/v0/I.png}
    \caption*{图2.2.2 异常值处理前I的volume和close的重叠k线图}
  \end{subfigure}
  \hfill
  \begin{subfigure}[t]{0.4\textwidth}
    \includegraphics[width=\textwidth]{./v2/v0/JM.png}
    \caption*{图2.2.2 异常值处理前JM的volume和close的重叠k线图}
  \end{subfigure}
  % 可以根据需要继续添加更多的子figure
  % \caption{整体图标题} % 如果需要为整个figure添加一个标题
\end{figure}
\newpage
\item 给出异常值处理后的close随时间变化的数值k线图:\FloatBarrier\noindent\begin{figure}[H]
  \centering
  \begin{subfigure}[t]{0.4\textwidth}
    \includegraphics[width=\textwidth]{./v2/v2/HC.png}
    \caption*{图2.2.1 异常值处理后HC的close的k线图}
  \end{subfigure}
  \hfill
  \begin{subfigure}[t]{0.4\textwidth}
    \includegraphics[width=\textwidth]{./v2/v2/I.png}
    \caption*{图2.2.2 异常值处理后I的close的k线图}
  \end{subfigure}
  \hfill
  \begin{subfigure}[t]{0.4\textwidth}
    \includegraphics[width=\textwidth]{./v2/v2/JM.png}
    \caption*{图2.2.2 异常值处理后JM的close的k线图}
  \end{subfigure}
  % 可以根据需要继续添加更多的子figure
  % \caption{整体图标题} % 如果需要为整个figure添加一个标题
\end{figure}
\item 给出异常值处理后的volume对比图:\FloatBarrier\noindent\begin{figure}[H]
  \centering
  \begin{subfigure}[t]{0.4\textwidth}
    \includegraphics[width=\textwidth]{./v2/v3/HC.png}
    \caption*{图2.2.1 异常值处理后HC的v2异常后close的k线图}
  \end{subfigure}
  \hfill
  \begin{subfigure}[t]{0.4\textwidth}
    \includegraphics[width=\textwidth]{./v2/v3/I.png}
    \caption*{图2.2.2 异常值处理后I的v2异常后close的k线图}
  \end{subfigure}
  \hfill
  \begin{subfigure}[t]{0.4\textwidth}
    \includegraphics[width=\textwidth]{./v2/v3/JM.png}
    \caption*{图2.2.2 异常值处理后JM的v2异常后close的k线图}
  \end{subfigure}
  % 可以根据需要继续添加更多的子figure
  % \caption{整体图标题} % 如果需要为整个figure添加一个标题
\end{figure}
\end{enumerate}
\fi

\newpage
\subsection{特征提取}
提取和涨跌幅强相关的参数:

\begin{enumerate}
    \item 交易量随时间的变化率
    \item 交易量 volume 滞后 30 分钟和滞后 1 天的特征
    \item 交叉特征(波动率 $\times$ 交易量)
\end{enumerate}

\subsubsection{交易量随时间的变化率}

\paragraph{金融学原理:}

交易量是市场活跃程度的重要指标。根据道氏理论和量价分析理论,价格变动若伴随交易量显著增长,则趋势更可能持续。交易量突增常伴随着主力资金的进出或情绪突变。

\paragraph{数学表达:}

交易量的变化率定义如下:
\[
\text{Volume Change Rate}_t = \frac{V_t - V_{t-1}}{V_{t-1}}
\]
其中 $V_t$ 表示当前时间点的交易量,$V_{t-1}$ 为上一个时间点的交易量。

\subsubsection{交易量滞后特征(30分钟和1天)}

\paragraph{金融学原理:}

滞后交易量可以反映市场在过去某一时刻的活跃程度,有助于捕捉市场的短期记忆效应与行为惯性。30分钟滞后反映了短周期的交易节奏,1天滞后则体现了日内波动对次日走势的影响。

\paragraph{数学表达:}

令 $k$ 为滞后周期,则有:
\[
\text{Lagged Volume}_{t-k} = V_{t-k}
\]
常用的周期为 $k = 30\text{min},\ 1\text{d}$。

也可构造其相对变化:
\[
\Delta V_{t,k} = V_t - V_{t-k}, \quad \text{或} \quad \frac{V_t - V_{t-k}}{V_{t-k}}
\]

\subsubsection{交叉特征(波动率 $\times$ 交易量)}

\paragraph{金融学原理:}

波动率衡量市场的不确定性,交易量反映市场的活跃度。两者的交叉特征可揭示市场剧烈波动前的征兆——当波动率与交易量同时升高,市场更可能出现大行情。

\paragraph{数学表达:}

首先定义过去 $n$ 个时间点的波动率为:
\[
\text{Volatility}_t = \sqrt{\frac{1}{n} \sum_{i=t-n+1}^{t} (P_i - \bar{P})^2}
\]
其中 $P_i$ 表示第 $i$ 个时间点的价格,$\bar{P}$ 为该窗口内的平均价格。

交叉特征则为:
\[
\text{Volume-Volatility Interaction}_t = \text{Volatility}_t \times V_t
\]

\newpage
\subsection{模型选择理由}

这是典型的监督学习 + 时间序列回归问题,特点是:

1.时间依赖性(前后时刻相关)

2.非线性特征影响(价格、成交量等复杂组合影响未来走势)\\

LSTM模型具备以下优势,利于实现这个任务:

1.记住较远历史信息

2.输入序列长度较长

3.输出依赖时间模式

4.输入输出为不定长序列\\

LSTM模型的数学-金融学背景如下:

\begin{itemize}
  \item[•] \textbf{门控机制的市场意义} \\
  LSTM 的门控机制天然适合捕捉期货市场的三类关键特征:

  \begin{table}[h]
  \centering
  \caption{LSTM门控与市场特征的对应关系}
  \begin{tabular}{lll}
  \toprule
  门控类型 & 数学表达 & 市场功能 \\
  \midrule
  遗忘门 & $\mathbf{f}_t=\sigma(\mathbf{W}_f[\mathbf{h}_{t-1},\mathbf{x}_t])$ & 过滤过时的技术指标 \\
   & & 衰减历史波动率影响 \\
  输入门 & $\mathbf{i}_t=\sigma(\mathbf{W}_i[\mathbf{h}_{t-1},\mathbf{x}_t])$ & 识别突破性行情 \\
   & & 吸收突发新闻事件 \\
  输出门 & $\mathbf{o}_t=\sigma(\mathbf{W}_o[\mathbf{h}_{t-1},\mathbf{x}_t])$ & 控制预测信号强度 \\
   & & 调节风险暴露程度 \\
  \bottomrule
  \end{tabular}
  \end{table}

  \item[•] \textbf{期货特征工程} \\
  输入特征设计为 5 维向量:
  \begin{equation*}
  \mathbf{x}_t = \begin{bmatrix}
  \frac{p_t - p_{t-5}}{p_{t-5}} & \text{(5分钟收益率)} \\
  \frac{\text{std}(p_{t-30:t})}{\text{mean}(p_{t-30:t})} & \text{(波动率)} \\
  \log(v_t/\bar{v}_{t-60}) & \text{(成交量偏离)} \\
  oi_t - oi_{t-30} & \text{(持仓量变化)} \\
  \mathbb{I}_{\text{夜盘时段}} & \text{(交易时段标记)}
  \end{bmatrix}
  \end{equation*}

  \item[•] \textbf{梯度传播的市场解释}
  \begin{center}
  \begin{minipage}[t]{0.6\linewidth}
  \begin{equation*}
  \frac{\partial \mathcal{L}}{\partial \mathbf{C}_{t-1}} = \mathbf{f}_t + \frac{\partial}{\partial \mathbf{C}_{t-1}}(\mathbf{i}_t \odot \tilde{\mathbf{C}}_t)
  \end{equation*}
  \end{minipage}
  \begin{minipage}[t]{0.35\linewidth}
  \begin{itemize}
    \item 趋势市中 $\mathbf{f}_t \approx 1$
    \item 震荡市中 $\mathbf{i}_t$ 主导更新
    \item 极端行情时梯度爆炸抑制
  \end{itemize}
  \end{minipage}
  \end{center}

  \item[•] \textbf{改进的 Peephole 结构}
  \begin{equation*}
  \begin{aligned}
  \mathbf{f}_t &= \sigma\left(\mathbf{W}_f[\mathbf{C}_{t-1}, \Delta p_{t-1}] + \mathbf{b}_f\right) \\
  \mathbf{o}_t &= \sigma\left(\mathbf{W}_o[\mathbf{C}_t, \text{VIX}_t] + \mathbf{b}_o\right)
  \end{aligned}
  \end{equation*}
  \begin{itemize}
    \item 价格加速度 $\Delta p_{t-1}$ 增强趋势判断
    \item VIX 指数调节风险控制强度
  \end{itemize}
\end{itemize}




\newpage
\subsection{模型具体实现} 

模型实现结构图全览:

\begin{center}
\scalebox{0.85}{ % 整体缩放
\tikzset{
  block/.style={rectangle, draw, text centered, minimum height=1.8em, minimum width=3em, rounded corners, font=\small},
  oval/.style={ellipse, draw, minimum height=1.8em, minimum width=4em, text centered, font=\small},
  arrow/.style={-Stealth, thick, shorten >=2pt, shorten <=2pt},
  box/.style={draw, thick, dashed, inner sep=0.2cm, rounded corners, font=\footnotesize}
}

\begin{tikzpicture}[node distance=0.8cm and 0.6cm]

% --- Preprocess Block ---
\node[oval] (data) {Data};

\node[block, below left=0.6cm and 0.5cm of data] (scaler) {MinMaxScaler};
\node[block, below=0.6cm of data] (cleaner) {Cleaner};
\node[block, below right=0.6cm and 0.5cm of data] (dataset) {TS Dataset};

\draw[arrow] (data) -- (scaler);
\draw[arrow] (data) -- (cleaner);
\draw[arrow] (data) -- (dataset);

\node[box, fit=(scaler)(cleaner)(dataset), label=below:Preprocess] (prebox) {};

% --- LSTM Module ---
\node[oval, below=1.2cm of cleaner] (lstm) {LSTM};
\draw[arrow] (cleaner) -- (lstm);

% Input Model
\node[oval, left=2.2cm of lstm] (input_model) {Input};
\node[block, below=0.6cm of input_model] (bn1) {BN};

\draw[arrow] (lstm) -- (input_model);
\draw[arrow] (input_model) -- (bn1);

% LSTM Layers
\node[oval, below=1cm of lstm] (lstmer) {LSTMer};
\node[block, below=0.6cm of lstmer] (lstm1) {LSTM64};
\node[block, below=0.6cm of lstm1] (drop1) {Drop0.3};
\node[block, below=0.6cm of drop1] (lstm2) {LSTM32};
\node[block, below=0.6cm of lstm2] (drop2) {Drop0.3};

\draw[arrow] (lstm) -- (lstmer);
\draw[arrow] (lstmer) -- (lstm1);
\draw[arrow] (lstm1) -- (drop1);
\draw[arrow] (drop1) -- (lstm2);
\draw[arrow] (lstm2) -- (drop2);

% Output Model
\node[oval, right=2.2cm of lstm] (output_model) {Output};
\node[block, below=0.6cm of output_model] (linear1) {Linear16};
\node[block, below=0.6cm of linear1] (bn2) {BN};
\node[block, below=0.6cm of bn2] (relu) {ReLU};
\node[block, below=0.6cm of relu] (drop3) {Drop0.15};

\draw[arrow] (lstm) -- (output_model);
\draw[arrow] (output_model) -- (linear1);
\draw[arrow] (linear1) -- (bn2);
\draw[arrow] (bn2) -- (relu);
\draw[arrow] (relu) -- (drop3);

% Merge Output
\node[block, below=1.5cm of drop2] (linear2) {Linear1};
\node[oval, below=0.6cm of linear2] (output_final) {Output};

\draw[arrow] (drop2) -- (linear2);
\draw[arrow] (drop3) -- (linear2);
\draw[arrow] (linear2) -- (output_final);

\node[box, fit=(input_model)(bn1)(lstmer)(drop2)(output_model)(drop3), label=below:Network] (netbox) {};
\end{tikzpicture}
}
\end{center}


模型实现结构分层次剖析:

\subsubsection{数据预处理}
\begin{itemize}
  \item[1.] \textbf{数据标准化} \\
  采用 Min-Max 标准化处理原始期货数据:
  \begin{equation}
  x_{\text{scaled}} = \frac{x - \min(X)}{\max(X) - \min(X)} \quad \text{(将特征缩放到 [0,1] 区间)}
  \end{equation}

  \item[2.] \textbf{数据清洗}
  \begin{itemize}
    \item 处理缺失值:前向填充(Forward Fill)
    \item 异常值处理:剔除 $\pm 3\sigma$ 外的价格数据
    \item 跳空修复:对隔夜缺口进行线性插值
  \end{itemize}

  \item[3.] \textbf{时序数据集构建} \\
  构建监督学习格式的时序数据:
  \begin{equation}
  \mathcal{D} = \{ (\mathbf{X}_t, y_t) \mid \mathbf{X}_t = [\mathbf{x}_{t-T}, ..., \mathbf{x}_t], \ y_t = p_{t+30} \}
  \end{equation}
  其中 $T=120$ 表示 2 小时历史窗口(对应图中 TimeSeries Dataset)
\end{itemize}


\subsubsection{网络架构}

\begin{itemize}
\item \textbf{输入特征}:包含交易量变化率,交易量滞后特征,交叉特征3个维度,经标准化处理后输入网络

\item \textbf{核心组件}:
  \begin{itemize}
  \item \textbf{BatchNorm层}:对输入特征进行归一化,设置$\epsilon=10^{-5}$防止数值不稳定
  \item \textbf{双层LSTM结构}:
    \begin{itemize}
    \item 第一层:64维隐藏状态,捕捉短期波动模式
    \item 第二层:32维隐藏状态,提取高阶时序特征
    \end{itemize}
  \item \textbf{Dropout层}:
    \begin{itemize}
    \item 第一层丢弃率0.3,缓解市场噪声影响
    \item 第二层丢弃率0.15,保留有效特征
    \end{itemize}
  \end{itemize}

\item \textbf{输出层设计}:
  \begin{itemize}
  \item 通过16维全连接层压缩特征
  \item ReLU激活保证预测收益率非负
  \item L2正则化($\lambda=0.01$)控制模型复杂度
  \end{itemize}
\end{itemize}


\newpage
\subsection{模型的训练与验证}
\subsubsection{数据预处理流程}
\begin{itemize}
\item \textbf{异常值处理}:
  \begin{itemize}
  \item 采用前向填充与线性插值组合策略:
    \[
    x_t^{\text{filled}} = \begin{cases}
    x_{t-1} & \text{单点缺失} \\
    \text{linear}(x_{t-k}, x_{t+m}) & \text{连续缺失}
    \end{cases}
    \]
  \item 剔除$\pm 3\sigma$外的极端值,保留市场正常波动范围
  \end{itemize}
  
\item \textbf{特征工程}:
  \begin{itemize}
  \item 动态缩放:对每个特征列独立进行MinMax标准化
    \[
    x^{(j)} \leftarrow \frac{x^{(j)} - \min(X^{(j)})}{\max(X^{(j)}) - \min(X^{(j)})}
    \]
  \item 状态保存:持久化scaler参数供生产环境复用
  \end{itemize}
\end{itemize}


\subsubsection{优化训练策略}
\begin{figure}[h]
\centering
\begin{tikzpicture}[node distance=1cm, auto]
\node (train) {训练epoch};
\node (loss) [right=of train] {计算复合损失};
\node (clip) [right=of loss] {梯度裁剪};
\node (update) [right=of clip] {参数更新};
\node (val) [below=of update] {验证评估};
\node (lr) [left=of val] {学习率调整};
\node (stop) [left=of lr] {早停判断};
\path[->] 
(train) edge (loss)
(loss) edge (clip)
(clip) edge (update)
(update) edge (val)
(val) edge (lr)
(lr) edge (stop);
\end{tikzpicture}
\caption{训练循环控制流}
\end{figure}

关键组件:
\begin{itemize}
\item \textbf{自适应优化器}:采用AdamW(改进版Adam):
  \begin{itemize}
  \item $\beta_1=0.9,\ \beta_2=0.999$ 平衡短期动量与长期方差
  \item 解耦权重衰减实现更稳定的L2正则
  \end{itemize}
  
\item \textbf{动态学习率}:ReduceLROnPlateau调度器:
  \[
  \eta \leftarrow \begin{cases}
  0.5\eta & \text{Val MSE连续5轮$\nrightarrow$} \\
  \eta & \text{otherwise}
  \end{cases}
  \]
  
\item \textbf{损失函数}:
  \[
  \mathcal{L} = \underbrace{\frac{1}{N}\sum(y-\hat{y})^2}_{\text{MSE}} + \lambda \underbrace{\|\mathbf{W}\|_2^2}_{\text{L2}} + \alpha \underbrace{\|\mathbf{h}_t-\mathbf{h}_{t-1}\|_2^2}_{\text{状态平滑}}
  \]

\item \textbf{正则化设计}:采用多层次正则化策略提升模型鲁棒性,具体实现如下:

\begin{center}
\begin{table}[h]
\centering
\begin{tabular}{lll}
\toprule
\textbf{类型} & \textbf{实现方式} & \textbf{金融逻辑} \\
\midrule
L2权重衰减 & Adam优化器weight\_decay=0.01 & 抑制市场噪声导致的过拟合 \\
Dropout & LSTM1:0.3 → LSTM2:0.15 & 渐进式特征选择 \\
梯度裁剪 & $\|\nabla W\|_2 \leq 1.0$ & 防范极端行情梯度爆炸 \\
BatchNorm & 输入层标准化 & 统一量纲 \\
LayerNorm & 隐层状态归一化 & 稳定时序特征 \\
\bottomrule
\end{tabular}
\caption{正则化策略配置表}
\end{table}
\end{center}

关键技术细节:
\begin{itemize}
\item \textbf{L2正则化}:
  \[
  \mathcal{L}_{reg} = \lambda\sum w_i^2 \quad (\lambda=0.01)
  \]
  通过AdamW优化器实现解耦权重衰减
  
\item \textbf{Dropout策略}:
  \begin{itemize}
  \item 第一层LSTM后:0.3丢弃率过滤噪声
  \item 第二层LSTM后:0.15丢弃率保留有效特征
  \end{itemize}

\item \textbf{梯度约束}:
  \[
  \mathbf{g} \leftarrow \min\left(1.0, \frac{1.0}{\|\mathbf{g}\|_2}\right)\mathbf{g}
  \]
  特别应对交割月合约的剧烈波动

\item \textbf{归一化处理}:
  \begin{itemize}
  \item 输入层:BatchNorm1d处理原始价格
  \item 隐层:LayerNorm适应变长时间序列
  \end{itemize}
\end{itemize}


\item \textbf{SMA数据增强}:
  \begin{itemize}
  \item \textbf{多周期移动平均}:
    \begin{align*}
    \text{SMA}_{20} &= \frac{1}{20}\sum_{i=0}^{19} p_{t-i} \\
    \text{SMA}_{60} &= \frac{1}{60}\sum_{i=0}^{59} p_{t-i} \\
    \text{EMA}_{10} &= 0.18p_t + 0.82\text{EMA}_{10}(p_{t-1})
    \end{align*}
    
  \item \textbf{衍生特征构建}:
    \begin{equation*}
    \begin{aligned}
    \text{Price-SMA20 Ratio} &= \frac{p_t}{\text{SMA}_{20}} - 1 \\
    \text{SMA20-SMA60 Delta} &= \text{SMA}_{20} - \text{SMA}_{60} \\
    \text{EMA10 Slope} &= \text{EMA}_{10}(p_t) - \text{EMA}_{10}(p_{t-5})
    \end{aligned}
    \end{equation*}
    
  \item \textbf{金融逻辑}:
    \begin{itemize}
    \item 价格/均线比率识别超买超卖状态
    \item 双均线差值判断趋势强度
    \item EMA斜率捕捉短期动量变化
    \end{itemize}
  \end{itemize}
\end{itemize}

\subsubsection{验证与模型选择}
\begin{itemize}
\item \textbf{双重评估指标}:
  \begin{align*}
  \text{MSE} &= \frac{1}{N}\sum(y_i-\hat{y}_i)^2 \quad \text{(强调极端误差)} \\
  \text{MAE} &= \frac{1}{N}\sum|y_i-\hat{y}_i| \quad \text{(衡量平均偏差)}
  \end{align*}
  
\item \textbf{早停机制}:
  \begin{itemize}
  \item 基于验证集MSE的patience=10策略
    \item 保留最佳模型状态:选择验证集均方误差最小对应的参数 $\mathbf{W}_{\text{best}}$,即
        \[
        \mathbf{W}_{\text{best}} = \min_{\mathbf{W}} \text{MSE}_{\text{val}}(\mathbf{W})
        \]
  \end{itemize}
  
\item \textbf{交叉验证}:时序分割(TimeSeriesSplit)保持时间依赖
\end{itemize}

\subsubsection{实现优化亮点}
\begin{itemize}
\item \textbf{内存管理}:
  \begin{itemize}
  \item DataLoader的pin\_memory加速GPU数据传输
  \item 梯度累积支持超大batch size
  \end{itemize}
  
\item \textbf{可复现性}:
  \begin{itemize}
  \item 设置全局随机种子(seed=42)
  \item 确定性算法配置
  \end{itemize}
  
\item \textbf{生产部署}:
  \begin{itemize}
  \item 模型量化(FP16)提升推理速度
  \item 异常输入自动检测与恢复
  \end{itemize}
\end{itemize}

\subsubsection{模型训练和验证的结果}
使用回归问题中常见的MSE均方误差和MAE平均绝对误差作为评价标准

选择HC,JM,RB三种期货进行可视化如下图

HC和JM在训练轮次到达4轮左右时, 误差接近收敛;

RB的验证集平均绝对误差呈波动下降, 没有出现过拟合和欠拟合的情况, 效果最优






\newpage
\subsection{模型预测效果与改进建议}
先根据模型计算出预测的收盘价close和实际收盘价做对比, 可视化展示发现误差在可控范围内

计算得到最终所需的涨跌幅数据, 展示其中3种期货的可视化图片

\newpage
\section{源码与文档}

\end{document}

